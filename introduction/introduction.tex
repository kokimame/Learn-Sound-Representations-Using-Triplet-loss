\normallinespacing

\chapter{Introduction}
% Maybe in Introduction
%Before breaking down the state-of-the-art technologies in the relevant field, we provide examples of key inventions in this field. We call it "Information Society" and we live in a historic period called "Information Age" (also know as "Digital Age"). Information is what resolves our decisions whichever it is economic, industrial, political or personal. 

%Collection and communication of information has been crutial since the emergence of agriculture.

% As an introduction to our work, we begin with a historical context of our work, briefly going back to the prehistoric times, describing the emergence of need for information technology to glance at how inherently important to humanity such technology is. From that point, the exponential growth of information in terms of both quality and quantity resulted in the establishment of today's society, sometimes referred as \textit{Information Society}, in which the usage, creation, distribution of information is a significant activity. 

%Instead of looking through each of relevant technologies, the center of our attention always lie at multimedia, especially at audio.
%The aim of this chapter is to provide a perspective that navigate readers in the course of the technological development from earliest pioneering work to more complex state-of-the-art methods which we will discuss in the next chapter.
 
%The importance of information had gradually grown over the long history. 

% Human's need for information has relatively gradually evolved as they expanded their intellectual activity and population.
% At least by 40,000-60,000 years ago, in hunter-gathering age, people mostly relied on immediate and environmental clues for decision making and most lived day to day without keeping much information for the future. With the emergence of agriculture around 10,000 years ago, however, people started recording and sharing information such as weather to maximize the outcome (their crops) in the future. The activity of storing, retrieving, and distributing information saw the first acceleration since its origin when the Sumerian in Mesopotamia developed writing in about 3000 BC.

% Despite the constant growth of need for information over the history, the availability and the accessibility of information has greatly improved in the recent decades. ...

% But the term \textit{information technology} in its modern sense first appeared in 1958, around the time when the early digital computers were moving into mass production.

Sharing and communication of information is the key activity that drove technological, neurological, and cultural advances in humanity. With the emergence of agriculture around 10,000 years ago, people in a hunting-gathering lifestyle had first ever had a demand of recording and sharing information such as weather to maximize their crops in the future. Since then, the importance of information has only magnified as our society become more complex. As people demanded more information, the mean of information management has improved with some remarkable breakthroughs, such as Cuneiform, the first writing system developed by the Sumerian in about BC, and the Gutenberg's printing system which succeeded automating the production of books. 

% Why we have huge data today
Today, information is stored and shared online. The amount  of online data keeps growing exponentially. According to IBM Marketing Cloud study (10 Key Marketing Trends for 2017), 90\% of this data on the internet has been created since 2016. To give an idea of how much and rapidly data is created on the internet today, the following activities are recognized as popular contributors of the data growth: users on YouTube uploads 500 hours of new video each minute of every day and Instagram users upload over 100 million photos and videos every day.

The bidirectional nature of the internet is one of the main factors that immensely accelerate the amount of online information we can access today. The users of the internet are not only the consumers of the data as in the traditional mass media such as TV and radio but also can be the producers who are equally able to contribute to the increase of online data. This mutual connection with others that the internet makes sharing contents online emotionally satisfying and further encourages people to transfer data to the internet [The Psychology Of Content Sharing Online In 2020].

This activity of sharing and the resultant increase of the online data was only possible after a multitude of technological advancements which started from the late 20th century in the area of processors, computer memory, computer storage, and computer networks. Before 1950s, around the time when the early digital computers were moving into mass production, information was often privately held or distributed in an unidirectional way such as from a radio host to the listeners and a book publisher to the readers.

% Multimedia
Not only the data quantity, the quality of the data had diversified over the recent years. In the earliest days, the internet was almost entirely text-based world. This limitation imposed by the computing power at the time was rapidly and steadily resolved as explained. With the invention of efficient coding and standards of multimedia data, such as MP3 for digital audio and MP4 for videos, a wider variety of media became more available online.

% Searching is important
From the beginning of the consumer internet, it is intended to be accessed by a large number of people at any moment. It has to be essential that the users can access the data effectively. Therefore, the data available on the internet must be well-organized so that anyone who has an access can retrieve what they look for. This requirement gets even more challenging partially as the size of the data gets larger.

% % % Purpose
% What kind of things we want to gain
% - Automatic organization of sounds
% - Content-based retrieval
Efficiency of the data retrieval is the major concern for the engineers managing large multimedia databases. In recent years, there were a number of experiments and attempts made to allow for more efficient retrieval. Among those approaches, in this work, we focus on automatic organization of the search results and content-based data retrieval.

First, automatic organization of the search results allows for search that more coherent to the users. Today, the billions of internet users not only search in a massive multimedia databases every day. 
Users typically submit search queries that express a broad intent, which makes the system often return large result sets. A query from a user can be a single word, multiple words or a sentence. The search results are generally presented in a list of results and users examine a few samples of the results until they find what they are looking for. This process is often designed on a basis of various technical consideration, for example, how to organize data efficiently, how to precisely respond to diverse user requests with inconsistent set of vocabulary, or how to present the search results to users.


\newpage


