\normallinespacing

\chapter{Introduction \texttt{<in progress>}}
% Maybe in Introduction
%Before breaking down the state-of-the-art technologies in the relevant field, we provide examples of key inventions in this field. We call it "Information Society" and we live in a historic period called "Information Age" (also know as "Digital Age"). Information is what resolves our decisions whichever it is economic, industrial, political or personal. 

%Collection and communication of information has been crutial since the emergence of agriculture.

As an introduction to our work, we begin with a historical context of our work, briefly going back to the prehistoric times, describing the emergence of need for information technology to glance at how inherently important to humanity such technology is. From that point, the exponential growth of information in terms of both quality and quantity resulted in the establishment of today's society, sometimes referred as \textit{Information Society}, in which the usage, creation, distribution of information is a significant activity. Instead of looking through each of relevant technologies, the center of our attention always lie at multimedia, especially at audio. The aim of this chapter is to provide a perspective that navigate readers in the course of the technological development from earliest pioneering work to more complex state-of-the-art methods which we will discuss in the next chapter.
 
%The importance of information had gradually grown over the long history. 
Human's need for information has relatively gradually evolved as they expanded their intellectual activity and population.
At least by 40,000-60,000 years ago, in hunter-gathering age, people mostly relied on immediate and environmental clues for decision making and most lived day to day without keeping much information for the future. With the emergence of agriculture around 10,000 years ago, however, people started recording and sharing information such as weather to maximize the outcome (their crops) in the future. The activity of storing, retrieving, and distributing information saw the first acceleration since its origin when the Sumerian in Mesopotamia developed writing in about 3000 BC.

Despite the constant growth of need for information over the history, the availability and the accessibility of information has greatly improved in the recent decades. ...

But the term \textit{information technology} in its modern sense first appeared in 1958, around the time when the early digital computers were moving into mass production.

% Huge data today
The amount of online data is growing exponentially. According to IBM Marketing Cloud study, 90\% of this data on the internet has been created since 2016. To give an idea of how much and rapidly data is shared, shared, and stored on the internet today, the following activities are recognized as popular contributors of the data growth: users on YouTube uploads 500 hours of new video each minute of every day and Instragram users upload over 100 million photos and videos every day. 

% Quote from the Shallows
The internet differs from most of the mass media it replaces in an obvious and very important way: it's bidirectional. We can send messages through the network as well as receive them. The ability to exchange information online, to upload as well as download, is one of the main factors that immensely accelerated the amount of information we can access today.

Regarding the use of online data and its applications, users not only upload the data but also search in a massive multimedia databases including audio, speech, text, video, image, and their combinations.  
 Users typically submit search queries that express a broad intent, which makes the system often return large result sets. A query from a user can be a single word, multiple words or a sentence. The search results are generally presented in a list of results and users examine a few samples of the results until they find what they are looking for. This process is often designed on a basis of various technical consideration, for example, how to organize data efficiently, how to precisely respond to diverse user requests with inconsistent set of vocabulary, or how to present the search results to users.

Methods used in content-based retrieval of multimedia data on the internet usually consider a number of common problems. User generated data on multimedia sharing websites such as image tags on Flickr, video descriptions on YouTube are often noisy, unstructured and produced (or recorded) under different conditions.
\newpage


